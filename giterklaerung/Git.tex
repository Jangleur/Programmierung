\documentclass[reqno]{article}
\usepackage{amsmath}
\usepackage[ngerman]{babel}
\usepackage[utf8]{inputenc} %erlaubt Umlaute
\usepackage[T1]{fontenc} %macht Wörter mit Umlauten trennbar
\usepackage{mathptmx}

\begin{document}
  \begin{enumerate}
  	\item Lade den vorhandenen Ordner von der Seite runter. Öffne dazu die Konsole, gehe zum Desktop, und tippe
  	  \begin{verbatim}
  	    git clone https://jangleur@github.com/kraftk/NetSec.git
  	  \end{verbatim}
  	  Wahrscheinlich musst du nun deine Passwort eingeben. Falls das nicht funktionieren sollte frag mich, vlt passt da irgendwas nicht...
  	\item Du solltest jetzt einen Ordner "NetSec" \, auf dem Desktop haben, in dem sich schon diverse Dateien befinden. Wenn du an dem Projekt arbeiten möchtest, versichere dich vorher mit 
  	  \begin{verbatim}
  	    git pull
  	  \end{verbatim}
  	  dass du dich auf dem aktuellen Stand befindest.
  	\item Hast du etwas geändert, so musst du deine Änderungen natürlich wieder hochladen. Mit
  	  \begin{verbatim}
  	    git status
  	  \end{verbatim}
      bekommst du eine Übersicht darüber, welche Dateien du geändert hast, und welche du schon geaddet hast (siehe nächster Schritt). Mit
      \begin{verbatim}
        git add <FILE1> <FILE2>
      \end{verbatim}
      fügst du Dateien hinzu. Entsprechend löscht du Dateien mit
      \begin{verbatim}
        git rm <FILE1> <FILE2>
      \end{verbatim} 
      Sobald du alle Dateien, die du hochladen möchtest hinzugefügt hast, musst du zuerst einen Kommentar zu deinen Änderungen geben. Dazu
      \begin{verbatim}
        git commit (-a)
      \end{verbatim}
      Hier öffnet sich nun ein seltsames Fenster. Um darin zu schreiben tippst du \texttt{i}, dann deinen commit. Um das Fenster wieder zu verlassen, drückst du \texttt{ESC}, \texttt{:wq} (write \& quit). Danach lädst du sie mit
      \begin{verbatim}
        git push
      \end{verbatim}
      endgültig hoch. Falls du dich wunderst, dass dir nicht alle Dateien angezeigt werden - im \texttt{.gitignore} File (verstecktes File) stehen alle Files, die von git ignoriert werden. Schau da mal rein.
  \end{enumerate}
\end{document}