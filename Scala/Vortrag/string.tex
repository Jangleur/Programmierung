	\section[Section]{Beispiele referenzielle Transparenz}
	\begin{frame}[fragile]
		\frametitle{Beispiel für referenzielle Transparenz}
		%Strings sind in Scala wie in Java Immutable.\\
		%Ein modifizierter String ist wirklich ein neuer String. Der
		%alte String bleibt wie er ist.\\
		\begin{lstlisting}[language=bash]
scala> val x = "Hello, World"
x: java.lang.String = Hello, World 

		
scala> val r1 = x.reverse 
r1: java.lang.String = dlroW ,olleH 

		
scala> val r2 = x.reverse 
r2: java.lang.String = dlroW ,olleH 
\end{lstlisting}	
\end{frame}
	\begin{frame}[fragile]
		\frametitle{Beispiel für referenzielle Transparenz}
		%Die Ausgabe ist die gleiche. Also war x referenziell Transparent.
		\begin{lstlisting}[language=bash]
scala> val r1 = "Hello, World".reverse
r1: java.lang.String = dlroW ,olleH 

		
scala> val r2 = "Hello, World".reverse 
r2: String = dlroW ,olleH 
		
\end{lstlisting}
\end{frame}