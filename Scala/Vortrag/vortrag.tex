\documentclass[12pt,utf8]{beamer}


%Wichtige Standard Pakete!
\usepackage[utf8]{inputenc}
\usepackage[ngerman]{babel}
\usepackage{xcolor}
\usepackage{graphicx}
\usepackage{amsmath}
\usepackage{mathtools}
\usepackage{amssymb}
\usepackage{amsthm}
\usepackage{listings}
\usepackage{mathabx}
\usepackage{tikz,pgf}
\usepackage{tikz-cd}
\usepackage{pgf}
\usepackage{algorithm, algorithmic}
\renewcommand{\algorithmicrequire}{\textbf{Eingabe:}}
%\renewcommand{\algorithmicensure}{\textbf{Ausgabe:}}
\usetikzlibrary{matrix,arrows}
\usepackage[autostyle=true,german=quotes]{csquotes}
\usepackage[all]{xy}
\usepackage{stmaryrd}
\usepackage{framed}
\usepackage{bbm}
\usepackage{cite}
\usepackage{etoolbox}
\usepackage{hyperref}
\usepackage{enumitem}
\newcommand\mySlash[2]{\ensuremath{%
		\!\sideset{_#1}{\!^#2}{\mathop\backslash}}}
\apptocmd{\sloppy}{\hbadness 10000\relax}{}{}

%Nummerierung der Folien
\beamertemplatefootpagenumber 


%Programmiersprache festlegen
\definecolor{dkgreen}{rgb}{0,0.6,0}
\definecolor{gray}{rgb}{0.5,0.5,0.5}
\definecolor{mauve}{rgb}{0.58,0,0.82}

\lstdefinestyle{myScalastyle}{
  frame=tb,
  language=scala,
  aboveskip=3mm,
  belowskip=3mm,
  showstringspaces=false,
  columns=flexible,
  basicstyle={\small\ttfamily},
  numbers=left,
  numberstyle=\tiny\color{gray},
  keywordstyle=\color{blue},
  commentstyle=\color{dkgreen},
  stringstyle=\color{mauve},
  frame=single,
  breaklines=true,
  breakatwhitespace=true,
  tabsize=2,
}


% Definiere div-Operator äquivalent zu mod
\makeatletter
\newcommand*{\bdiv}{%
  \nonscript\mskip-\medmuskip\mkern5mu%
  \mathbin{\operator@font div}\penalty900\mkern5mu%
  \nonscript\mskip-\medmuskip
}
\newcommand{\testleftlong}{\longleftarrow\!\shortmid}

\makeatother
\usepackage{rotating}
\newcommand\tabrotate[1]{\begin{turn}{90}\rlap{#1}\end{turn}}
\newcommand{\invertediota}{\begin{sideways}%
		\begin{sideways}$\iota$\end{sideways}\end{sideways}}
% Benutze theoremstyles von Thomas, ein klein wenig verändert
\newtheoremstyle{break-italic}% name
  {1ex}%      Space above, empty = `usual value'
  {}%      Space below
  {\itshape}% Body font
  {0cm}%         Indent amount (empty = no indent, \parindent = para indent)
  {\bfseries}% Thm head font
  {}%        Punctuation after thm head
  {\newline }% Space after thm head: \newline = linebreak
  {}%         Thm head 
\newtheoremstyle{break-roman}% name
  {1ex}%      Space above, empty = `usual value'
  {}%      Space below
  {\normalfont}% Body font
  {}%         Indent amount (empty = no indent, \parindent = para indent)
  {\bfseries}% Thm head font
  {}%        Punctuation after thm head
  {\newline }% Space after thm head: \newline = linebreak
  {}%         Thm head spec
  
\theoremstyle{break-italic} 
\newtheorem{defe}[theorem]{Definition}

\newcommand{\einschraenkung}{\,\rule[-5pt]{0.4pt}{12pt}\,{}}




%Für den Header notwendig!
\usepackage[percent]{overpic}

%Sourcecode
\usepackage{listings}

%Einbinden des Themes
%\input{Theme/ubuntuuserstheme.tex}

%Mathe
\usepackage{amsthm}

%Standard Angaben
\author{Jan Albert}
\title{Funktionale Programmierung in Scala}
\date{\today}
%\titlegraphic{\includegraphics[scale=0.04]{book2.jpg}}

\begin{document}


	%Titelseite
	\begin{frame}
		\titlepage
	
	\end{frame}
	
	%Inhaltsverzeichnis
	\begin{frame}
		\frametitle{Inhaltsverzeichnis}
		\tableofcontents
	\end{frame}
	
	%Buch
	\input{buch}
	
	%Einfuehrung
		%Einfuehrung
	\section[Section]{Was ist Funktionale Programmierung?}
		\begin{frame}
			\frametitle{Was ist Funktionale Programmierung?}
				\underline{Idee:}
				Benutzt ausschließlich "reine Funktionen" d. h. Funktionen, welche 					keine Seiteneffekte haben. \\
				\leavevmode \\
				\underline{Beispiele für Seiteneffekte:}
				\begin{itemize}
				\item[•] Verändern/Modifizieren einer Variable
				\item[•] Verändern/Modifizieren einer Datenstruktur
				\item[•] Eine Exception werfen
				\item[•] Konsolen Eingabe/Ausgabe
				\item[•] Lesen/Schreiben aus/von einer Datei
				\end{itemize}
		\end{frame}
	
	%Definitionen
		\section[Section]{Reine Funktionen}
	\begin{frame}
		\frametitle{Definitionen}
		\begin{definition}[Reine Funktionen]
			Eine \emph{reine Funktion} mit Eingabetyp $A$ und Ausgabetyp $B$ (Schreibweise: $A				\Rightarrow B$)
			ist eine Berechnung, welche jeden Wert $a$ vom Typ $A$ genau einen Wert 			$b$ vom Typ $B$ zuordnet, sodass $b$ nur aus dem Wert von $a$ bestimmt 				wird.
		\end{definition} 
		\leavevmode \\
		\underline{Beispiele:}
		\begin{itemize}
		\item[•] Eine Funktion intToString vom Typ Int $\Rightarrow$ String bildet 
		jede ganze Zahl auf einen String ab und macht nichts anderes.
		\item[•] Die Addition von ganzen Zahlen. 
		\end{itemize}
	\end{frame}
	

	%Ausdruck
		\section*{}
		\begin{frame}
			\frametitle{Ausdruck}
			\begin{definition}
				Jeder Teil eines Programms, welcher zu einem Ergebnis 								zusammengefasst werden kann, d. h. alles was man in den 
				Scala-Interpreter tippt und ein Ergebnis liefert,
				nennt man einen \emph{Ausdruck}.
		\end{definition} 
		\leavevmode \\
		Beispiel:  $2+3$	
		\end{frame}
	
	%RT
	\section[Section]{Referenziell Transparent (RT)}
	\begin{frame}
		\frametitle{Referenziell Transparent (RT)}
		\begin{defe}[Referenziell Transparent (RT)]
			Ein Ausdruck $e$ ist Referenziell Transparent (RT),
			wenn für alle Programme $p$, alle 
		\end{defe} 
	
	%Beispiel1
		\section[Section]{Beispiel mit Seiteneffekt}
		\begin{frame}[fragile]
		\frametitle{Beispiel mit Seiteneffekt}
		\begin{lstlisting}[style=myScalastyle]
class Cafe {
	def buyCoffee(cc: CreditCard): Coffee = {
		val cup = new Coffee()
		cc.charge(cup.price)
		cup
	}
}
\end{lstlisting}
\end{frame}
		
			
	
	%Beispiel2
		\section[Section]{Beispiel ohne Seiteneffekt}
		\begin{frame}[fragile]
		\frametitle{Beispiel ohne Seiteneffekt}
		\begin{lstlisting}[style=myScalastyle]
class Cafe {
	def buyCoffee(cc: CreditCard): (Coffee, Charge) = {
		val cup = new Coffee()
		(cup, Charge(cc, cup.price))
	}
}
\end{lstlisting}
\end{frame}
	
	%Klasse Charge
	\input{charge}
	
	%Spielerei
	\section*{}
	\begin{frame}[fragile]
	\frametitle{RT im Beispiel}
	\begin{tcolorbox}[colback=blue!5,colframe=brightcerulean,title=RT im Beispiel]
		Der Returntype von cc.charge(cup.price) "verschwindet"\\
		in buyCoffee. Das Ergebnis von buyCoffee(aCreditCard) ist
		nur cup, was äquivalent zu new Coffee() ist. Wenn buyCoffee eine reine 				Funktion wäre, so müsste für jedes Programm p, sich 								p(buyCoffee(aCreditCard)) und p(new Coffee()) gleich verhalten.
	\end{tcolorbox}
\end{frame}

			
	
	%Quellen
	\section[Section]{Quellen}
	\begin{frame}
  		\frametitle{Quellen}    
  		\begin{thebibliography}{10}    
  		\beamertemplatebookbibitems
  		\bibitem{Autor1990}
    		Paul Chiusano,
    		Runar Bjarnason
   		 \newblock {\em Functional Programming in Scala}
    	\newblock Manning, 2014.
  		\beamertemplatearticlebibitems
  		\bibitem{Jemand2000}
    	S.~Jemand.
    	\newblock On this and that.
    	\newblock {\em Journal of This and That}, 2(1):		50--100, 2000.
  		\end{thebibliography}
	\end{frame}
	
	%Danksagung
	\section[Section]{Danksagung}
	\begin{frame}
		\vfill
		\begin{center}\begin{Huge}Vielen Dank \\[10pt]
			für eure Aufmerksamkeit.\end{Huge}\vfill
		\end{center}
		\vfill
	\end{frame}
	
\end{document}