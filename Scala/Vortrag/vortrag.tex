\documentclass[12pt,utf8]{beamer}


%Wichtige Standard Pakete!
\usepackage[ngerman]{babel}
\usepackage{xcolor}
\usepackage{graphicx}

% Dummy Text
\usepackage{lipsum}

%Für den Header notwendig!
\usepackage[percent]{overpic}

%Sourcecode
\usepackage{listings}

%Einbinden des Themes
%\input{Theme/ubuntuuserstheme.tex}

%Standard Angaben
\author{Jan Albert}
\title{Funktionale Programmierung in Scala}
\date{\today}

\begin{document}

	%Titelseite
	\begin{frame}
		\titlepage
	\end{frame}
	
	%Inhaltsverzeichnis
	\begin{frame}
		\frametitle{Inhaltsverzeichnis}
		\tableofcontents
	\end{frame}
	
	%Spielerei
	\section[Section]{Spielerei}
	\lstdefinestyle{customc}{
  		belowcaptionskip=1\baselineskip,
  		breaklines=true,
  		frame=L,
  		xleftmargin=\parindent,
  		language=C,
  		showstringspaces=false,
  		basicstyle=\footnotesize\ttfamily,
  		keywordstyle=\bfseries\color{green!40!black},
  		commentstyle=\itshape\color{purple!40!black},
  		identifierstyle=\color{blue},
  		stringstyle=\color{orange},
	}

	\lstdefinestyle{customasm}{
  		belowcaptionskip=1\baselineskip,
  		frame=L,
  		xleftmargin=\parindent,
  		language=[x86masm]Assembler,
  		basicstyle=\footnotesize\ttfamily,
  		commentstyle=\itshape\color{purple!40!black},
	}

	\lstset{escapechar=@,style=customc}   
	\begin{lstlisting}
		#include <stdio.h>
		#define N 10
		/* Block
 		* comment */

		int main()
		{
    		int i;

    		// Line comment.
    		puts("Hello world!");
    
    		for (i = 0; i < N; i++)
    		{
        		puts("LaTeX is also great for programmers!");
    		}

    		return 0;
		}
	\end{lstlisting}


	
	%Quellen
	\section[Section]{Quellen}
	\begin{frame}[allowframebreaks]
  		\frametitle<presentation>{Quellen}    
  		\begin{thebibliography}{10}    
  		\beamertemplatebookbibitems
  		\bibitem{Autor1990}
    		Paul Chiusano,
    		Runar Bjarnason
   		 \newblock {\em Functional Programming in Scala}
    	\newblock Manning, 2014.
  		\beamertemplatearticlebibitems
  		\bibitem{Jemand2000}
    	S.~Jemand.
    	\newblock On this and that.
    	\newblock {\em Journal of This and That}, 2(1):		50--100, 2000.
  		\end{thebibliography}
	\end{frame}
	
	%Danksagung
	\section[Section]{Danksagung}
	\begin{frame}
		\vfill
		\begin{center}\begin{Huge}Vielen Dank \\[10pt]
			für eure Aufmerksamkeit.\end{Huge}\vfill
		\end{center}
		\vfill
	\end{frame}
	
\end{document}